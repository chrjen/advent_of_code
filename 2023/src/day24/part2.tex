\documentclass[11pt]{article}

\usepackage{amsmath}
\usepackage{bm}
\usepackage{calc}
\usepackage{biblatex}
\usepackage{environ}         % provides \BODY
\usepackage{etoolbox}        % provides \ifdimcomp
\usepackage{graphicx}        % provides \resizebox
\usepackage{hyperref}
\usepackage[a4paper, margin=3cm]{geometry}

\newcommand{\uvecx}{{\mathbf{\hat{\textnormal{\bfseries x}}}}}
\newcommand{\uvecy}{{\mathbf{\hat{\textnormal{\bfseries y}}}}}
\newcommand{\uvecz}{{\mathbf{\hat{\textnormal{\bfseries z}}}}}


\begin{document}

\section*{Advent of Code 2023 – Day 24 Part 2}

In the problem all bodies move in a linear fashion which can be described
by the following vector equation. $\vec{s}(t)$ is the position after a time $t$,
$\vec{p}$ is the initial position at time 0, and $\vec{v}$ is the velocity.
\begin{equation}
    \vec{s}(t) = \vec{p} + \vec{v}t
\end{equation}

So we can start by finding when the stone $s$ and any hailstone are both in the same place
at the same time. We can write this as $\vec{s_s}(t) = \vec{s_0}(t)$ which when expanded gives
us the following equation.
\begin{align}
    \vec{p_s} + \vec{v_s}t = \vec{p_0} + \vec{v_0}t \\
    \vec{p_s} - \vec{p_0} + (\vec{v_s} - \vec{v_0})t = \vec{0}
\end{align}

To start solving this I took inspiration from this post on
Github\footnote{\url{https://gist.github.com/tom-huntington/00065d3f5c52a900bfa99a6230470956}}
making use of the wedge product to solve for the initial position. Wedge product is a
concept from Exterior algebra\footnote{\url{https://en.wikipedia.org/wiki/Exterior_algebra}} which
was new to me. It took me down the rabbit hole that was Geometric algebra, especially the YouTube
channel sudgylacmoe\footnote{\url{https://www.youtube.com/@sudgylacmoe}} was very helpful.

The first step is to wedge both sides with $(\vec{v_s} - \vec{v_0})$ which eliminates time
component since any vector wedged with itself is zero.
\begin{align}
    [\vec{p_s} - \vec{p_0} + (\vec{v_s} - \vec{v_0})t] \wedge (\vec{v_s} - \vec{v_0}) &= \vec{0} \wedge (\vec{v_s} - \vec{v_0}) \nonumber \\
    \vec{p_s} \wedge (\vec{v_s} - \vec{v_0}) - \vec{p_0} \wedge (\vec{v_s} - \vec{v_0})  &= \vec{0} \nonumber \\
    \vec{p_s} \wedge \vec{v_s} - \vec{p_s} \wedge \vec{v_0} - \vec{p_0} \wedge \vec{v_s} + \vec{p_0} \wedge \vec{v_0}  &= \vec{0} \nonumber 
\end{align}

Next we want to also eliminate the stone's initial velocity $\vec{v}_s$ from the equation. To do this
we can use the equation above, but for a different hailstone with initial position and velocity $\vec{p}_1$
and $\vec{v}_1$. Then subtract that new equation from the equation above for hailstone 0. This eliminates
the constant $\vec{p_s} \wedge \vec{v_s}$ term shared between them.
\begin{gather*}
    (\vec{p_s} \wedge \vec{v_s} - \vec{p_s} \wedge \vec{v_0} - \vec{p_0} \wedge \vec{v_s} + \vec{p_0} \wedge \vec{v_0}) - (\vec{p_s} \wedge \vec{v_s} - \vec{p_s} \wedge \vec{v_1} - \vec{p_1} \wedge \vec{v_s} + \vec{p_1} \wedge \vec{v_1}) = \vec{0} \\
    \vec{p_s} \wedge \vec{v_s} - \vec{p_s} \wedge \vec{v_0} - \vec{p_0} \wedge \vec{v_s} + \vec{p_0} \wedge \vec{v_0} - \vec{p_s} \wedge \vec{v_s} + \vec{p_s} \wedge \vec{v_1} + \vec{p_1} \wedge \vec{v_s} - \vec{p_1} \wedge \vec{v_1} = \vec{0} \\
    - \vec{p_s} \wedge \vec{v_0} - \vec{p_0} \wedge \vec{v_s} + \vec{p_0} \wedge \vec{v_0} + \vec{p_s} \wedge \vec{v_1} + \vec{p_1} \wedge \vec{v_s} - \vec{p_1} \wedge \vec{v_1} = \vec{0}
\end{gather*}

Further rearranging and grouping of term we can get the following.
\begin{gather*}
    \vec{p_s} \wedge \vec{v_1} - \vec{p_s} \wedge \vec{v_0} + \vec{p_1} \wedge \vec{v_s} - \vec{p_0} \wedge \vec{v_s} + \vec{p_0} \wedge \vec{v_0} - \vec{p_1} \wedge \vec{v_1} = \vec{0} \\
    \vec{p_s} \wedge (\vec{v_1} - \vec{v_0}) + (\vec{p_1} - \vec{p_0}) \wedge \vec{v_s} + \vec{p_0} \wedge \vec{v_0} - \vec{p_1} \wedge \vec{v_1} = \vec{0}
\end{gather*}

Then to finally get eliminate $\vec{v}_s$ from the equation we wedge both sides with
$(\vec{p_1} - \vec{p_0})$. 
\begin{gather*}
    (\vec{p_1} - \vec{p_0}) \wedge [\vec{p_s} \wedge (\vec{v_1} - \vec{v_0}) + (\vec{p_1} - \vec{p_0}) \wedge \vec{v_s} + \vec{p_0} \wedge \vec{v_0} - \vec{p_1} \wedge \vec{v_1}] = \vec{0} \\
    (\vec{p_1} - \vec{p_0}) \wedge \vec{p_s} \wedge (\vec{v_1} - \vec{v_0}) +  (\vec{p_1} - \vec{p_0}) \wedge \vec{p_0} \wedge \vec{v_0} -  (\vec{p_1} - \vec{p_0}) \wedge \vec{p_1} \wedge \vec{v_1} = \vec{0} \\
    (\vec{p_1} - \vec{p_0}) \wedge \vec{p_s} \wedge (\vec{v_1} - \vec{v_0}) +  \vec{p_1} \wedge \vec{p_0} \wedge \vec{v_0} + \vec{p_0} \wedge \vec{p_1} \wedge \vec{v_1} = \vec{0}
\end{gather*}

Lastly we can swap the order of the wedge product and subtraction, and group terms together to get
the final equation below. An interesting thing is that until this point we have made no assumption
about the dimension of our vector, meaning that this should be valid in 2D as well as 5D.
\begin{align} \label{final_wedge}
    \vec{p_s} \wedge ( \vec{p_0} - \vec{p_1} ) \wedge ( \vec{v_0} - \vec{v_1} ) + \
    \vec{p_0} \wedge \vec{p_1} \wedge ( \vec{v_0} - \vec{v_1} ) = \vec{0}
\end{align}

Next we want to expand the wedge product. Expanding any three vectors with three dimensions
gives us the following.
\begin{align}
    (a_0\uvecx + b_0\uvecy + c_0\uvecz) \wedge (a_1\uvecx + b_1\uvecy + c_1\uvecz) \wedge (a_2\uvecx + b_2\uvecy + c_2\uvecz) \nonumber \\
    = ( a_0(b_1c_2 - c_1b_2) + b_0(c_1a_2 - a_1c_2) + c_0(a_1b_2 - b_1a_2) ) \uvecx \wedge \uvecy \wedge \uvecz  \label{wedge_partial_expand} \\
    = ( a_0b_1c_2 - a_0c_1b_2 + b_0c_1a_2 - b_0a_1c_2 + c_0a_1b_2 - c_0b_1a_2 ) \uvecx \wedge \uvecy \wedge \uvecz \label{wedge_full_expand}
\end{align}

Now we can expand equation \eqref{final_wedge} using formula \eqref{wedge_partial_expand} and
\eqref{wedge_full_expand} and moving one of the terms over gives us the following equation.
\begin{align}
    ( &p_{sx}((p_{0y}-p_{1y})(v_{0z}-v_{1z}) - (p_{0z}-p_{1z})(v_{0y}-v_{1y})) + \nonumber \\
    &p_{sy}((p_{0z}-p_{1z})(v_{0x}-v_{1x}) - (p_{0x}-p_{1x})(v_{0z}-v_{1z})) + \nonumber \\
    &p_{sz}((p_{0x}-p_{1x})(v_{0y}-v_{1y}) - (p_{0y}-p_{1y})(v_{0x}-v_{1x})) ) \uvecx \wedge \uvecy \wedge \uvecz \nonumber \\
    &= - ( p_{0x}p_{1y}(v_{0z}-v_{1z}) - p_{0x}p_{1z}(v_{0y}-v_{1y}) + p_{0y}p_{1z}(v_{0x}-v_{1x}) - \nonumber \\
    &\quad p_{0y}p_{1x}(v_{0z}-v_{1z}) + p_{0z}p_{1x}(v_{0y}-v_{1y}) - p_{0z}p_{1y}(v_{0x}-v_{1x}) ) \uvecx \wedge \uvecy \wedge \uvecz
\end{align}

To simplify the differences are replaced with a combined subscripts, in order. For example
the difference $(p_{0y}-p_{1y})$ will be rewritten as $p_{01y}$. I also remove the minus
from the right hand side.
\begin{align}
      &( p_{sx}(p_{01y}v_{01z} - p_{01z}v_{01y}) + p_{sy}(p_{01z}v_{01x} - p_{01x}v_{01z}) + p_{sz}(p_{01x}v_{01y} - p_{01y}v_{01x}) ) \uvecx \wedge \uvecy \wedge \uvecz \nonumber \\
    = &(-p_{0x}p_{1y}v_{01z} + p_{0x}p_{1z}v_{01y} - p_{0y}p_{1z}v_{01x} + p_{0y}p_{1x}v_{01z} - p_{0z}p_{1x}v_{01y} + p_{0z}p_{1y}v_{01x} ) \uvecx \wedge \uvecy \wedge \uvecz \nonumber
\end{align}

We can remove $\uvecx \wedge \uvecy \wedge \uvecz$ from both sides and we get an equation with three unknowns.
If we include two more such equations for different hailstones we get a system of three linear equations
and three unknowns, which we can solve.
\begin{align}
    & p_{sx}(p_{01y}v_{01z} - p_{01z}v_{01y}) + p_{sy}(p_{01z}v_{01x} - p_{01x}v_{01z}) + p_{sz}(p_{01x}v_{01y} - p_{01y}v_{01x}) \nonumber \\
  = &-p_{0x}p_{1y}v_{01z} + p_{0x}p_{1z}v_{01y} - p_{0y}p_{1z}v_{01x} + p_{0y}p_{1x}v_{01z} - p_{0z}p_{1x}v_{01y} + p_{0z}p_{1y}v_{01x} \nonumber \\
    & p_{sx}(p_{12y}v_{12z} - p_{12z}v_{12y}) + p_{sy}(p_{12z}v_{12x} - p_{12x}v_{12z}) + p_{sz}(p_{12x}v_{12y} - p_{12y}v_{12x}) \nonumber \\
  = &-p_{1x}p_{2y}v_{12z} + p_{1x}p_{2z}v_{12y} - p_{1y}p_{2z}v_{12x} + p_{1y}p_{2x}v_{12z} - p_{1z}p_{2x}v_{12y} + p_{1z}p_{2y}v_{12x} \nonumber \\
    & p_{sx}(p_{20y}v_{20z} - p_{20z}v_{20y}) + p_{sy}(p_{20z}v_{20x} - p_{20x}v_{20z}) + p_{sz}(p_{20x}v_{20y} - p_{20y}v_{20x}) \nonumber \\
  = &-p_{2x}p_{0y}v_{20z} + p_{2x}p_{0z}v_{20y} - p_{2y}p_{0z}v_{20x} + p_{2y}p_{0x}v_{20z} - p_{2z}p_{0x}v_{20y} + p_{2z}p_{0y}v_{20x} \nonumber
\end{align}

Let's rewrite it as a matrix equation which we then can give to a computer to have it solve it using
something like Gauss-Jordan Elimination.
\begin{align}
    &\begin{bmatrix}
        p_{01y}v_{01z} - p_{01z}v_{01y} & p_{01z}v_{01x} - p_{01x}v_{01z} & p_{01x}v_{01y} - p_{01y}v_{01x} \\
        p_{12y}v_{12z} - p_{12z}v_{12y} & p_{12z}v_{12x} - p_{12x}v_{12z} & p_{12x}v_{12y} - p_{12y}v_{12x} \\
        p_{20y}v_{20z} - p_{20z}v_{20y} & p_{20z}v_{20x} - p_{20x}v_{20z} & p_{20x}v_{20y} - p_{20y}v_{20x}
    \end{bmatrix}
    \begin{bmatrix}
        p_{sx} \\
        p_{sy} \\
        p_{sz}
    \end{bmatrix} \nonumber \\
    =
    &\begin{bmatrix}
        -p_{0x}p_{1y}v_{01z} + p_{0x}p_{1z}v_{01y} - p_{0y}p_{1z}v_{01x} + p_{0y}p_{1x}v_{01z} - p_{0z}p_{1x}v_{01y} + p_{0z}p_{1y}v_{01x} \\
        -p_{1x}p_{2y}v_{12z} + p_{1x}p_{2z}v_{12y} - p_{1y}p_{2z}v_{12x} + p_{1y}p_{2x}v_{12z} - p_{1z}p_{2x}v_{12y} + p_{1z}p_{2y}v_{12x} \\
        -p_{2x}p_{0y}v_{20z} + p_{2x}p_{0z}v_{20y} - p_{2y}p_{0z}v_{20x} + p_{2y}p_{0x}v_{20z} - p_{2z}p_{0x}v_{20y} + p_{2z}p_{0y}v_{20x} \nonumber
    \end{bmatrix}
\end{align}

\end{document}